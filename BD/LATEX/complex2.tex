\documentclass{article}

% Commande pour définir les métadonnées
\newcommand{\setMeta}[2]{}

\setMeta{title}{Complexité algorithmique}
\setMeta{difficulty}{moyen}
\setMeta{solution}{0} 
\setMeta{bonus}{0}
\setMeta{figures}{hagrid/harrypotter}
\setMeta{author}{Moi}
\setMeta{language}{français}

\begin{document}

\section{Exercice}

\subsection*{Question}
Considérez l'algorithme suivant qui recherche un élément donné dans un tableau trié par recherche binaire :
\begin{verbatim}
fonction recherche_binaire(tab, x):
    début = 0
    fin = longueur(tab) - 1
    tant que début <= fin faire
        milieu = (début + fin) / 2
        si tab[milieu] == x alors
            retourner vrai
        sinon si tab[milieu] < x alors
            début = milieu + 1
        sinon
            fin = milieu - 1
        fin si
    fin tant que
    retourner faux
\end{verbatim}

Quelle est la complexité temporelle de cet algorithme ? Justifiez votre réponse.

\end{document}
